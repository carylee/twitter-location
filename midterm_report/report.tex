\documentclass[12pt]{article}
\usepackage[left = 3cm,top=2cm,right=3cm,nohead, letterpaper]{geometry}                
\usepackage[parfill]{parskip}    % Activate to begin paragraphs with an empty line rather than an indent
\usepackage{graphicx}
\usepackage{amssymb}
\usepackage{amsmath}
\usepackage{epstopdf}
\usepackage{algorithmic}
\usepackage{verbatim}
\DeclareGraphicsRule{.tif}{png}{.png}{`convert #1 `dirname #1`/`basename #1 .tif`.png}
\graphicspath{{/Users/Loch/Dropbox/algorithms/}}

\title{EECS 395 \\ Project Update}
\author{Kai Hayashi, Cary Lee, \& Dan Myers}
\date{}                                           % Activate to display a given date or no date

\begin{document}
\maketitle
\section*{Introduction:}
Twitter represents a vast collection of human interaction. With its easy data collection and free and open nature, it represents a treasure trove of information for the well-equipped scientist. Recently, Twitter added a feature allowing users to keep track of their location via their phone's GPS coordinates. We are interested in the following question: ``Can someone determine a tweeter's location based off of the content of the tweet alone?" This question has been a hot topic in the news as of late, with reported robberies and many concerns over the public data available in tweets. The GPS information combined with twitter's large user base, may provide enough data to build a distribution of user locations given the tweet data. In this report we present our method of gathering twitter data and provide some preliminary results of a very basic attempt at solving this problem. 

\section*{Data:}


\section*{Preliminary Results:}


\end{document}